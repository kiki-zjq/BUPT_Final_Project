\documentclass[11pt]{article}
\usepackage[utf8]{inputenc}
\usepackage[document]{ragged2e}
\usepackage{fancyhdr}
\usepackage{lastpage}
% \title{EBU6304}
% \author{zhujiangqi0701 }
% \date{January 2021}
\usepackage{geometry}
\usepackage{color}

\geometry{
	a4paper,
	left=20mm,
	top=20mm,
	right=20mm,
}


\begin{document}
	
	% \maketitle
	
	\thispagestyle{empty}
	\textbf{\Huge EBU6304}  \textcolor[rgb]{0.5,0.5,0.5}{\normalsize COVID-19 Alternative Assessment (Even Sem - Paper A)}\\
	~\\~\\~\\~\\
	
	\textbf{\Large Joint Programme Assessments 2019/20}
	~\\
	~\\
	\textbf{\large EBU6304 Software Engineering}
	~\\
	~\\
	
	\underline{Answering this paper requires \textbf{2 hours}}; Answers to be submitted within the allocated \textbf{6 hours} window.
	
	~\\
	~\\
	\textbf{\large Answer ALL questions}\\
	~\\
	
	\fbox{%
		\parbox[c][150 pt][c]{480 pt}{
			
			\textbf{\normalsize \hspace{10pt} \underline{INSTRUCTIONS}}\\
			
			\begin{enumerate}
				\item \textbf{You must NOT share any content from this document during the assessment period.}
				\item Your answers must be typed, and diagrams or equations must be written clearly and legibly with black or blue colour \textbf{and in English}.
				\item You need to submit your answers BEFORE the allocated deadline.
				\item \textbf{Read the instructions on the inside cover of the questions sheet.}
			\end{enumerate}
			
		}%
	}\\
	~\\~\\~\\
	\textbf{\large Examiners}
	
	\hspace{2em}Dr.Yiding
	~\\~\\~\\
	\textbf{\footnotesize Copyright  Beijing University of Posts and Telecommunications \&  Queen Mary University of London 2020}\\
	~\\
	Filename: YidingEstimateFile
	
	%%%%%%%%%%%%%%%%%%%%%%%%%%%%%%%%%%%%%%%%%%%%%%%%%%%%%%%%%%%%%%%
	
	\newpage
	\thispagestyle{empty}
	\newgeometry{
		left=20mm,
		top=5mm,
		right=20mm,
	}
	\part*{{\Large Instructions}}
	% \section{Introduction}
	~\\
	This is an open-book assessment, which should be completed \textbf{within 2 hours}. You MUST submit your answers within 6 hours from the assessment being released.\\
	~\\
	
	You MUST complete the assessment on your own, without consulting any other person. You MAY NOT check your answers with any other person.\\
	~\\
	
	You can refer to textbooks, notes and online materials to facilitate your working, if you provide a direct quote, or copy a diagram or chart, you must cite the source.\\
	~\\
	
	\textbf{\underline{Before you start the assessment}}
	\begin{enumerate}
		\item[1) ] Read the questions thoroughly and understand them
		\item[2) ] Ensure you have all the resources you require to complete and upload the final assessment.
		\item[3) ] If you require any assistance, \textbf{raise the issue via the messaging section of this assessment on QMPlus}, immediately.
	\end{enumerate}
	~\\
	
	\textbf{\underline{During the assessment session}}
	\begin{enumerate}
		\item[1) ] Use the supplied answer sheet document to enter your answers. Start on a new page for each question. Make sure it is clear which question number you are answering.
		\item[2) ] \textbf{\underline{Type your answers}} in the supplied answer sheet; hand-written equations or sketches can be incorporated into the answer sheet. Please save your work at least every 15 minutes so that you do not 
		risk losing it.
		\item[3) ] When completed answering all questions, perform a word count and list the number of words on the answer sheet, then save the file as pdf before uploading, \textbf{only pdf will be accepted}, any other file format will not be accepted.
		\item[4) ] Your submission must be your own work, and you must ensure that you do not break any of the rules in the Academic Misconduct Policy.
	\end{enumerate}
	~\\
	
	\textbf{\underline{Submitting the Assessment}}
	\begin{enumerate}
		\item[1) ] You will have 6 hours from the start of the scheduled assessment time – do not leave submissions too close to the deadline. \textbf{NO late submission will be accepted, no exceptions}.
		\item[2) ] Make sure you upload and submit the final version before the deadline.
		\item[3) ] Please be aware that submissions will be subject to review, including but not limited to plagiarism detection software.
	\end{enumerate}
	~\\
	
	If you have any problems relating to access or submitting during the assessment period, please contact 
	the email (it-issues@qmbupt.org), state the module code in the subject, and clearly state your name 
	and student ID and any issues you are experiencing. You must use either @qmul.ac.uk or 
	@bupt.edu.cn email address. Requests from external email addresses will not be processed.
	
	%%%%%%%%%%%%%%%%%%%%%%%%%%%%%%%%%%%%%%%%%%%%%%%%%%%%%%%%%%%%%%%
	
	\newpage
	
	\pagestyle{fancy}
	\fancyhf{}
	\rhead{\large{2019/20}}
	\lhead{\large{EBU6304\_ALT2020\_A}}
	\rfoot{\large Page \textbf{\thepage} of \textbf{\pageref{unknown}}}
	
	\setcounter{page}{1}
	\newgeometry{
		left=20mm,
		top=20mm,
		right=20mm,
	}
	
	
        \textbf{Question 1}

        Question 1
        \begin{enumerate}
            
        \item[a) ]Discuss why sometimes it is sensible to deliver an incomplete software product and then issue new versions after delivery. Give a specific example of such a software product you have used during the COVID-19 pandemic, and describe how the software product adapted to the rapid changing situation with incremental delivery.
        \begin{flushright}
            \textbf{[9 marks]}
        \end{flushright}
    
        \item[b) ]The COVID-19 crisis has forced many company employees to work from home. What are the impacts of working from home on an Agile software development team? What can the team do to minimise the impacts? Discuss TWO areas of impact and propose your solutions.
        \begin{flushright}
            \textbf{[16 marks]}
        \end{flushright}
    
        \end{enumerate}
        ~\\
    
        \textbf{Question 2}

          A start-up company has decided to develop an online streaming platform. You are a member of their Agile software development team and you are aware that there are some big competitors in the field (e.g., Netflix). You are worried because your team leader seems reluctant to focus on the architectural aspects of the system. Your worries increase when your team leader sketches the following architecture on the whiteboard during a team meeting
        \begin{enumerate}
            
        \item[a) ]Highlight the importance and advantages of focusing more on the system architecture;
        \begin{flushright}
            \textbf{[5 marks]}
        \end{flushright}
    
        \item[b) ]Your team plans to follow an agile software development approach, explain if this will be achievable or not while putting more emphasis on developing the architecture;
        \begin{flushright}
            \textbf{[6 marks]}
        \end{flushright}
    
        \item[c) ]Explain the limits of the simple architectural model sketched in Figure 1, make reference to the problem scenario;
        \begin{flushright}
            \textbf{[6 marks]}
        \end{flushright}
    
        \item[d) ]Propose modifications that can be made to this simple architectural model to cope better with the problem scenario. Sketch the result with a diagram and discuss it.
        \begin{flushright}
            \textbf{[8 marks]}
        \end{flushright}
    
        \end{enumerate}
        ~\\
    
        \textbf{Question 3}

          The company mentioned in the above question 2 launched the online streaming platform. It has been very successful until a similar new product with better features was introduced into the market by its competitor. The company has to improve its product to include more features to catch up with the market, but unfortunately the chief software engineer resigned to join its competitor. The company has no immediate replacement for her as she was the only senior software engineer in the company.
        \begin{enumerate}
            
        \item[a) ]Identify and discuss TWO types of risks involved in the start-up company.
        \begin{flushright}
            \textbf{[5 marks]}
        \end{flushright}
    
        \item[b) ]Discuss what the start-up company failed to do which could have prevented the problem they are facing.
        \begin{flushright}
            \textbf{[8 marks]}
        \end{flushright}
    
        \item[c) ]Provide ONE short-term and ONE long-term advice to the start-up company on how to get back into the market quickly.
        \begin{flushright}
            \textbf{[12 marks]}
        \end{flushright}
    
        \end{enumerate}
        ~\\
    
        \textbf{Question 4}

          Question 4
        \begin{enumerate}
            
        \item[a) ]An important aspect of good quality programming is to keep a clear distinction between what an object as defined by a class does, and how it does it. Explain the meaning of the statement and why it is important.
        \begin{flushright}
            \textbf{[8 marks]}
        \end{flushright}
    
        \item[b) ]Explain how the decorator design pattern works, and why it is a good technique for modifying the way an object works.
        \begin{flushright}
            \textbf{[8 marks]}
        \end{flushright}
    
        \item[c) ]Write a class FailStorer that uses the decorator design pattern so that any time the method move is called and returns false, the Position that was its argument is added to a set of Positions. The class must also have a method getFailPositions() that returns the set of Positions created, and resets the set to store Positions to an empty set.
        \begin{flushright}
            \textbf{[9 marks]}
        \end{flushright}
    
        \end{enumerate}
        ~\\
    
	
	
	~\\~\\~\\
	\begin{center}
		END OF PAPER
	\end{center}
	
	
	\label{unknown}
\end{document}
