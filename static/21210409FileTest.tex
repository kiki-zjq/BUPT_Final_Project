\documentclass[11pt]{article}
\usepackage[utf8]{inputenc}
\usepackage[document]{ragged2e}
\usepackage{fancyhdr}
\usepackage{lastpage}
% \title{EBU6304}
% \author{zhujiangqi0701 }
% \date{January 2021}
\usepackage{geometry}
\usepackage{color}

\geometry{
	a4paper,
	left=20mm,
	top=20mm,
	right=20mm,
}


\begin{document}
	
	% \maketitle
	
	\thispagestyle{empty}
	\textbf{\Huge EBU6304}  \textcolor[rgb]{0.5,0.5,0.5}{\normalsize COVID-19 Alternative Assessment (Even Sem - Paper A)}\\
	~\\~\\~\\~\\
	
	\textbf{\Large Joint Programme Assessments 2020/21}
	~\\
	~\\
	\textbf{\large EBU6304 SE}
	~\\
	~\\
	
	\underline{Answering this paper requires \textbf{2 hours}}; Answers to be submitted within the allocated \textbf{6 hours} window.
	
	~\\
	~\\
	\textbf{\large Answer ALL questions}\\
	~\\
	
	\fbox{%
		\parbox[c][150 pt][c]{480 pt}{
			
			\textbf{\normalsize \hspace{10pt} \underline{INSTRUCTIONS}}\\
			
			\begin{enumerate}
				\item \textbf{You must NOT share any content from this document during the assessment period.}
				\item Your answers must be typed, and diagrams or equations must be written clearly and legibly with black or blue colour \textbf{and in English}.
				\item You need to submit your answers BEFORE the allocated deadline.
				\item \textbf{Read the instructions on the inside cover of the questions sheet.}
			\end{enumerate}
			
		}%
	}\\
	~\\~\\~\\
	\textbf{\large Examiners}
	
	\hspace{2em}Dr.Jiangqi Dr.Jifei Dr.Yi Ding
	~\\~\\~\\
	\textbf{\footnotesize Copyright  Beijing University of Posts and Telecommunications \&  Queen Mary University of London 2020}\\
	~\\
	Filename: 21210409FileTest
	
	%%%%%%%%%%%%%%%%%%%%%%%%%%%%%%%%%%%%%%%%%%%%%%%%%%%%%%%%%%%%%%%
	
	\newpage
	\thispagestyle{empty}
	\newgeometry{
		left=20mm,
		top=5mm,
		right=20mm,
	}
	\part*{{\Large Instructions}}
	% \section{Introduction}
	~\\
	This is an open-book assessment, which should be completed \textbf{within 2 hours}. You MUST submit your answers within 6 hours from the assessment being released.\\
	~\\
	
	You MUST complete the assessment on your own, without consulting any other person. You MAY NOT check your answers with any other person.\\
	~\\
	
	You can refer to textbooks, notes and online materials to facilitate your working, if you provide a direct quote, or copy a diagram or chart, you must cite the source.\\
	~\\
	
	\textbf{\underline{Before you start the assessment}}
	\begin{enumerate}
		\item[1) ] Read the questions thoroughly and understand them
		\item[2) ] Ensure you have all the resources you require to complete and upload the final assessment.
		\item[3) ] If you require any assistance, \textbf{raise the issue via the messaging section of this assessment on QMPlus}, immediately.
	\end{enumerate}
	~\\
	
	\textbf{\underline{During the assessment session}}
	\begin{enumerate}
		\item[1) ] Use the supplied answer sheet document to enter your answers. Start on a new page for each question. Make sure it is clear which question number you are answering.
		\item[2) ] \textbf{\underline{Type your answers}} in the supplied answer sheet; hand-written equations or sketches can be incorporated into the answer sheet. Please save your work at least every 15 minutes so that you do not 
		risk losing it.
		\item[3) ] When completed answering all questions, perform a word count and list the number of words on the answer sheet, then save the file as pdf before uploading, \textbf{only pdf will be accepted}, any other file format will not be accepted.
		\item[4) ] Your submission must be your own work, and you must ensure that you do not break any of the rules in the Academic Misconduct Policy.
	\end{enumerate}
	~\\
	
	\textbf{\underline{Submitting the Assessment}}
	\begin{enumerate}
		\item[1) ] You will have 6 hours from the start of the scheduled assessment time – do not leave submissions too close to the deadline. \textbf{NO late submission will be accepted, no exceptions}.
		\item[2) ] Make sure you upload and submit the final version before the deadline.
		\item[3) ] Please be aware that submissions will be subject to review, including but not limited to plagiarism detection software.
	\end{enumerate}
	~\\
	
	If you have any problems relating to access or submitting during the assessment period, please contact 
	the email (it-issues@qmbupt.org), state the module code in the subject, and clearly state your name 
	and student ID and any issues you are experiencing. You must use either @qmul.ac.uk or 
	@bupt.edu.cn email address. Requests from external email addresses will not be processed.
	
	%%%%%%%%%%%%%%%%%%%%%%%%%%%%%%%%%%%%%%%%%%%%%%%%%%%%%%%%%%%%%%%
	
	\newpage
	
	\pagestyle{fancy}
	\fancyhf{}
	\rhead{\large{2019/20}}
	\lhead{\large{EBU6304\_ALT2020\_A}}
	\rfoot{\large Page \textbf{\thepage} of \textbf{\pageref{unknown}}}
	
	\setcounter{page}{1}
	\newgeometry{
		left=20mm,
		top=20mm,
		right=20mm,
	}
	
	
        \textbf{Question 1}

        This is a question
        \begin{enumerate}
            
        \item[a) ]This is a subquestion
        \begin{flushright}
            \textbf{[5 marks]}
        \end{flushright}

        \textcolor{red}{\textbf{Criteria:}}
    
        \setlength{\parindent}{2em}\textcolor{red}{1. This is mark criteria 1 ... ...\textbf{[2 marks]}}
    
        \setlength{\parindent}{2em}\textcolor{red}{2. This is mark criteria 2 ... ...\textbf{[3 marks]}}
    
        \item[b) ]This is a subquestion 2
        \begin{flushright}
            \textbf{[5 marks]}
        \end{flushright}

        \textcolor{red}{\textbf{Criteria:}}
    
        \setlength{\parindent}{2em}\textcolor{red}{1. This is mark criteria 3 ... ...\textbf{[2 marks]}}
    
        \setlength{\parindent}{2em}\textcolor{red}{2. This is mark criteria 4 ... ...\textbf{[2 marks]}}
    
        \setlength{\parindent}{2em}\textcolor{red}{3. This is mark criteria 5 ... ...\textbf{[1 marks]}}
    
        \end{enumerate}
        ~\\
    
	
	
	~\\~\\~\\
	\begin{center}
		END OF PAPER
	\end{center}
	
	
	\label{unknown}
\end{document}
